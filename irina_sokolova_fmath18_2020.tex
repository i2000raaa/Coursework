\documentclass[12pt]{book}
%\usepackage[cp1251]{inputenc}
\usepackage[english,russian]{babel}
\usepackage{amssymb,latexsym,amsmath,euscript,amsfonts,amsthm}\usepackage{graphicx}
\usepackage{color}

\usepackage{fontspec}
\usepackage{polyglossia}
\setdefaultlanguage{russian}
\setmainfont[Mapping=tex-text]{CMU Serif}
\usepackage[font=small]{caption}

\title{Курсовая работа}
\author{Соколова Ирина}

\newtheorem{theorem}{Теорема}[chapter]
\newtheorem{lemma}{Лемма}[chapter]
\newtheorem*{corollary}{Следствие}
\newtheorem{definition}{Определение}[chapter]
\newtheorem*{conjecture}{Гипотеза}
\newtheorem*{example}{Пример}
\newtheorem*{remark}{Замечание}
\newtheorem*{criteria}{Критерий}

\newcommand{\RomanNumeralCaps}[1]
    {\MakeUppercase{\romannumeral #1}}

\newcounter{exercise}
\newenvironment{exercise}{\refstepcounter{exercise}\smallskip\noindent\textbf{Упражнение~\theexercise.}}{}
\numberwithin{exercise}{chapter}

\newenvironment{demo}{\noindent\textsl{Доказательство. }}{ $\square$ \medskip}

\begin{document}
\pagestyle{plain}
\makeatletter

\section{Введение}

Введение

        На \RomanNumeralCaps{2} Международном Конгрессе математиков в Париже в 1900 году немецким математиком, Д. Гильбертом, были сформулированы 23 математические проблемы, которые предстояло решить в \RomanNumeralCaps{20} веке. Одной из этих проблем, а именно шестнадцатой, является «Проблема топологии алгебраических кривых и поверхностей». Шестнадцатая проблема Д. Гильберта разбивается на две части, первая из которых – это исследование топологии и расположения неособых алгебраических кривых на проективной плоскости RP2 и неособых алгебраических поверхностей в проективном пространстве RP3. Кроме того, данная проблема подразумевает исследование топологии вещественных алгебраических многообразий размерности d в вещественном проективном пространстве RPq, где q \geqslant 3, 1 \leqslant d \leqslant (q–1), а также исследование вещественных алгебраических многообразий с особенностями. 

%%\@input{preface}
%%\@input{ch1_1}
%%\@input{bibliography}
\makeatother
\end{document}
